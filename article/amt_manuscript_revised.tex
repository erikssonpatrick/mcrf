%from% Copernicus Publications Manuscript Preparation Template for LaTeX Submissions
%% ---------------------------------
%% This template should be used for copernicus.cls
%% The class file and some style files are bundled in the Copernicus Latex Package, which can be downloaded from the different journal webpages.
%% For further assistance please contact Copernicus Publications at: production@copernicus.org
%% https://publications.copernicus.org/for_authors/manuscript_preparation.html


%% Please use the following documentclass and journal abbreviations for discussion papers and final revised papers.

%% 2-column papers and discussion papers
\documentclass[journal abbreviation, manuscript]{copernicus}



%% Journal abbreviations (please use the same for discussion papers and final revised papers)


% Advances in Geosciences (adgeo)
% Advances in Radio Science (ars)
% Advances in Science and Research (asr)
% Advances in Statistical Climatology, Meteorology and Oceanography (ascmo)
% Annales Geophysicae (angeo)
% Archives Animal Breeding (aab)
% ASTRA Proceedings (ap)
% Atmospheric Chemistry and Physics (acp)
% Atmospheric Measurement Techniques (amt)
% Biogeosciences (bg)
% Climate of the Past (cp)
% DEUQUA Special Publications (deuquasp)
% Drinking Water Engineering and Science (dwes)
% Earth Surface Dynamics (esurf)
% Earth System Dynamics (esd)
% Earth System Science Data (essd)
% E&G Quaternary Science Journal (egqsj)
% Fossil Record (fr)
% Geochronology (gchron)
% Geographica Helvetica (gh)
% Geoscience Communication (gc)
% Geoscientific Instrumentation, Methods and Data Systems (gi)
% Geoscientific Model Development (gmd)
% History of Geo- and Space Sciences (hgss)
% Hydrology and Earth System Sciences (hess)
% Journal of Micropalaeontology (jm)
% Journal of Sensors and Sensor Systems (jsss)
% Mechanical Sciences (ms)
% Natural Hazards and Earth System Sciences (nhess)
% Nonlinear Processes in Geophysics (npg)
% Ocean Science (os)
% Primate Biology (pb)
% Proceedings of the International Association of Hydrological Sciences (piahs)
% Scientific Drilling (sd)
% SOIL (soil)
% Solid Earth (se)
% The Cryosphere (tc)
% Web Ecology (we)
% Wind Energy Science (wes)


%% \usepackage commands included in the copernicus.cls:
%\usepackage[german, english]{babel}
%\usepackage{tabularx}
%\usepackage{cancel}
%\usepackage{multirow}
%\usepackage{supertabular}
%\usepackage{algorithmic}
%\usepackage{algorithm}
%\usepackage{amsthm}
%\usepackage{float}
%\usepackage{subfig}
%\usepackage{rotating}


\begin{document}

\title{Synergistic radar and radiometer retrievals of ice hydrometeors}

% \Author[affil]{given_name}{surname}

\Author[1]{Simon}{Pfreundschuh}
\Author[1]{Patrick}{Eriksson}
\Author[2]{Stefan A.}{Buehler}
\Author[2]{Manfred}{Brath}
\Author[4]{David}{Duncan}
\Author[3]{Richard}{Larsson}
\Author[1]{Robin}{Ekelund}

\affil[1]{Department of Space, Earth and Environment, Chalmers University of Technology, 41296 Gothenburg, Sweden}
\affil[2]{Meteorologisches Institut, Fachbereich Geowissenschaften, Centrum für Erdsystem und Nachhaltigkeitsforschung (CEN), Universität Hamburg, Bundesstraße 55, 20146 Hamburg, Germany}
\affil[3]{Max Planck Institute for Solar System Research, Justus-von-Liebig-Weg 3, 37077 Göttingen, Germany}
\affil[4]{European Centre for Medium-Range Weather Forecasts, Shinfield Park, Reading RG2 9AX, United Kingdom}
%% The [] brackets identify the author with the corresponding affiliation. 1, 2, 3, etc. should be inserted.

\runningtitle{Retrieving frozen hydrometeors from combined radar and sub-millimeter observations}
\runningauthor{Simon Pfreundschuh}
\correspondence{Simon Pfreundschuh (simon.pfreundschuh@chalmers.se)}

\received{}
\pubdiscuss{} %% only important for two-stage journals
\revised{}
\accepted{}
\published{}

%% These dates will be inserted by Copernicus Publications during the typesetting process.

\firstpage{1}

\maketitle

\begin{abstract}
  The upcoming Ice Cloud Imager (ICI) radiometer, to be launched on board the
  second generation of European operational meteorological satellites
  (Metop-SG), will be the first microwave imager to provide sub-millimeter
  observations of the atmosphere. The Microwave Imager (MWI) radiometer will be
  flown on the same satellites and complement the ICI sensor with observations
  at traditional millimeter wavelengths. The addition of these two new passive
  microwave sensors to the global system of earth observation satellites opens
  up opportunities for synergistic satellite missions aiming to maximize
  the scientific return of the Metop-SG program. This study analyzes the
  potential benefits of combining observations of the MWI and ICI radiometers
  with a 94-$\unit{GHz}$ cloud radar for the retrieval of frozen hydrometeors.
  Starting from a simplified numerical experiment, it is shown that the
  complementary information content in the radar and radiometer observations can
  help to better constrain the particle size distribution of ice particles in
  the atmosphere. The feasibility of the combined retrieval is demonstrated by
  applying a one-dimensional, variational cloud-retrieval algorithm to simulated
  observations from a high-resolution atmospheric model. Comparison of the
  results with passive- and radar-only versions of the retrieval algorithm
  confirms that synergies between the active and passive observations allow an
  improved retrieval of microphysical properties of frozen hydrometeors. The
  effect of the assumed ice particle shape on the results is analyzed and found
  to be critical for obtaining good retrieval performance. In addition to this,
  the synergistic retrieval shows improved sensitivity to liquid water in both
  warm and supercooled clouds. The results of this study clearly demonstrate the
  potential of the combined observations to constrain the microphysical
  properties of ice hydrometeors, which can help to reduce errors in retrieved
  profiles of mass- and number densities.\end{abstract}


\introduction  %% \introduction[modified heading if necessary]

Ice clouds play an important role in many weather- and climate-related processes
in the atmosphere. They interact with incoming and outgoing radiation and thus
influence the Earth's energy budget. Moreover, as part of the global
hydrological cycle and due to their relation to the dynamics of the atmosphere
\citep{bony15}, observations of ice clouds provide important information to
constrain the state of the atmosphere in numerical weather prediction (NWP) models
\citep{geer} as well as to validate predictions from climate models
\citep{waliser09}.

Despite the importance of observations of ice clouds for climate and weather
prediction, today's global observing system cannot provide accurate information
on the global distribution of ice in the atmosphere
\citep{eliasson11,duncan18a}. The main difficulty in sensing atmospheric ice
from space is the large variability of sizes and concentrations in which ice
particles occur in the atmosphere. The wide spectrum of ice crystal sizes, which
ranges from micro- to millimeter scales, can only be partially resolved by
currently available space-borne sensors.

The sensitivity of a remote sensing system to ice particles of a given size is
determined mainly by its observing frequencies. The scattering of radiation by
ice particles is strongest for sizes roughly equal to the wavelength, $\lambda$,
of the radiation. For particles with sizes much smaller than $\lambda$, the
sensitivity decreases rapidly, making them practically invisible to the sensor.
Although the strength of the interaction between particles and radiation
decreases as the wavelength becomes much larger than the particle size, it
remains strong enough for the cloud signal to saturate in the presence of
thicker clouds, leading to loss of sensitivity further down the line of sight.

Currently available systems for observing ice hydrometeors typically make use of
radiation from the microwave, infrared or optical domain. Infrared and optical
sensors provide sensitivity to small ice particles but cannot sense significant
parts of the ice mass of thicker clouds due to saturation of the signal.
Microwave observations, in contrast, provide sensitivity throughout the whole
atmospheric column but are insensitive to small ice particles. Although radars
and lidars generally provide greater sensitivity than their passive
counterparts, they are ultimately limited by the same principles.

To narrow the size-sensitivity gap between the infrared and traditional microwave
sensors, the upcoming Ice Cloud Imager (ICI) will extend the microwave
frequencies available for studying clouds with channels at $243$, $325$, $448$ and
$664\ \unit{GHz}$ \citep{Eriksson19}. This extension of the smallest currently available microwave
wavelength from $1.6\ \unit{mm}$ at $183\ \unit{GHz}$ down to the sub-millimeter
domain ($0.45\ \unit{mm}$ at $664\ \unit{GHz}$) will significantly improve the
size-sensitivity of space-borne microwave observations of clouds.

Together with ICI, the newly developed Microwave Imager (MWI) will be flown on
the satellites of the Metop-SG program. MWI will complement ICI's observations
with measurements at traditional millimeter wavelengths as well as a novel
spectral band around the $118\ \unit{GHz}$ Oxygen line. The observations of MWI,
which cover the frequency range from $19\ \unit{GHz}$ up to $183\ \unit{GHz}$,
will provide additional sensitivity to liquid and frozen precipitation as well
as water vapor.

The advent of space-borne sub-millimeter radiometry of clouds brings with it
great potential for the study of ice in the atmosphere. The information content and
retrieval performance of radiometer observations alone has been studied in
detail for column-integrated ice mass \citep{jimenez07, wang17, brath18a} as
well as for the vertical distribution of ice in the atmosphere \citep{birman17, grutzun18,
  aires19}. Also the concept of combining millimeter and sub-millimeter
radiometer observations with active observations from a cloud radar has been
investigated \citep{evans05, jiang19}.

Combined retrievals using radar and passive radiometer observations, have also
been developed for the Tropical Rainfall Measuring Mission (TRMM,
\citet{kummerow98, grecu04}) and the Global Precipitation Measurement (GPM,
\cite{hou14, grecu16, munchak11}) mission. However, since the principal target
of these missions were liquid hydrometeors, they make use of sensors at
comparably low microwave frequencies, which provide only limited sensitivity to
frozen hydrometeors.

This work applies the concept of synergistic radar and sub-millimeter radiometer
retrievals to the upcoming ICI and MWI sensors by combining them with a
conceptual, nadir-pointing W-band cloud radar. It extends previous studies on
this observational technique by providing an in-depth analysis of the
fundamental synergies between the active and passive observations that help to
improve the retrieval ice in the atmosphere. In particular, this study
investigates to which extent the combined active and passive observations can
constrain the microphysics of ice particles in the atmosphere. Starting from a
simplified numerical experiment, the complementarity of the information content
of the active and passive observations is demonstrated. In addition to this,
simulated results from a synergistic, variational cloud-retrieval algorithm are
presented. The algorithm is applied to synthetic observations of cloud scenes
from a cloud-resolving atmospheric model and used to further explore the
synergies between the active and passive observations.

The presented research has been conducted as part of a larger study funded by
the European Space Agency, which evaluated the concept of a future radar mission
to fly in constellation with ICI on board the satellites of the Metop-SG
program. Inspired by the concept of the Global Precipitation Measurement (GPM,
\cite{hou14}) mission, the approach of this tentative mission is to perform
vertically-resolved, high-accuracy retrievals of hydrometeors from the
co-located active and passive observations at the swath center of the passive
imager. The results of combined retrieval could then be used to constrain
passive-only profile retrievals with the aim of extending the profiling
capabilities of the radar to the wide swath of the passive imager.

Following this introduction, Section \ref{sec:methods_and_data} introduces the
test data, sensor configuration and the developed retrieval algorithm on which
the study is based. This is followed by the experimental results on the
information content of the combined observations and the retrieval results of
the joint retrieval on selected test scenes in Section \ref{sec:results}. The
article closes with a discussion of the results in Section \ref{sec:discussion}
and conclusions in Section \ref{sec:conclusions}.


\section{Methods and data}
\label{sec:methods_and_data}

The synergistic retrieval is tested using simulated observations of cloud scenes
from a cloud-resolving atmospheric circulation model. This section presents the
selected reference cloud scenes, sensor configuration and basic modeling
assumptions used in the radiative transfer simulations. In addition to this, the
theoretical formulation of the combined cloud-retrieval algorithm is introduced.

\subsection{Reference cloud scenes}

The cloud scenes that will be used for the testing of the retrieval were
produced by Environment and Climate Change Canada using a high-resolution NWP
configuration of the Global Environmental Multiscale (GEM) Model
(\cite{cote98}). For this study, we restrict ourselves to two designated,
two-dimensional test scenes, which are displayed in Fig.~\ref{fig:overview}. The
test scenes have a horizontal resolution of $1\ \unit{km}$ and both extend over
$800\ \unit{km}$. The scenes were chosen with the aim of covering a large range
of cloud structures and compositions so as to ensure a realistic assessment of
the retrieval. The first test scene, shown in panel (a), is located in the
tropical Pacific and contains a convective storm system in the northern half of
the scene and its anvil that extends into the southern half of the scene. The
second scene, shown in panel (b), is located in the North Atlantic and contains
an ice cloud in the southern part and a low-level, mixed-phase cloud in the
remainder of the scene.

\begin{figure}[h!]
\centering
\includegraphics[width = 0.8\textwidth]{figures/fig01.png}
\caption{The distribution of total hydrometeor mass content in the two
cloud scenes used to test the retrieval. Colored lines show the
 $m = 10^{-5}\  \unit{kg\ m^{-3}}$ contour for different
 hydrometeor species.}
\label{fig:overview}
\end{figure}


The GEM model uses six types of hydrometeors to represent clouds and
precipitation \citep{milbrandtyau05}: Two classes of liquid hydrometeors (rain
and liquid cloud) and four of frozen hydrometeors (cloud ice, snow, hail and
graupel). The particle size distribution (PSD) of each hydrometeor type is
parametrized by its particle number concentration and mass density. The full
particle size distribution can be prognosed from the two moments using a
species-dependent parametrization and mass-size relationship. The parameters of
the mass-size relationship are given in Tab.~\ref{tab:species_parameters}. As
shown in the table, the masses of all ice particles in the model are
assumed to scale with a power of three, which leads to high densities for
large particles.

\begin{table}
  \centering
  \caption{Particle-model names, IDs and parameters $\alpha, \beta$ of the
    mass-size relationships $m = \alpha D_\text{max}^\beta$, where
    $D_\text{max}$ is the maximum diameter of the particle. The ID column
    contains the particle shape identifier of the particle model in the
    \citet{eriksson18} scattering database.}
  \label{tab:species_parameters}
  \begin{tabular}{l|c|c|c|c}
    Hydrometeor species & Particle shape & ID & $\alpha$ & $\beta$ \\
    \hline
    Cloud ice    & GemCloudIce  & 31 & 440   & 3 \\
    Snow         & GemSnow      & 32 & 52.4  & 3 \\
    Graupel      & GemGraupel   & 33 & 209.4 & 3 \\
    Hail         & GemHail      & 34 & 471.2 & 3 \\
    Rain         & LiquidSphere & 25 & 523.6 & 3 \\
    Liquid cloud & LiquidSphere & 25 & 523.6 & 3 \\
  \end{tabular}
\end{table}

Examples of particle size distributions of frozen hydrometeors are displayed in
Fig.~\ref{fig:gem_psds}. The four panels display the prognosed particle size
distributions for the four frozen hydrometeor types together with renderings of
the particle shapes used in the forward simulations. As these plots show, the
assumed particle size distributions across different ice species vary mostly in
their horizontal and vertical scaling, whereas the function shape shows less
variability. Furthermore, an important characteristic of the model can be
identified here, which will help to better understand the retrieval results
presented later: Cloud ice in the model is characterized by high particle number
densities and small particle sizes, whereas snow exhibits lower number
concentrations and larger particles.


\begin{figure}[h!]
\centering \includegraphics[width = \textwidth]{figures/fig02.png}
\caption{Realizations of particle size distributions from the cloud scenes used
  in this study. The number particle density is plotted with respect to the
  volume-equivalent diameter $D_\text{eq}$. Shown are the PSDs corresponding to
  100 randomly chosen grid points with a mass concentration higher than
  $10^{-6}\ \unit{kg\ m^{-3}}$. Line color encodes the corresponding mass density.}
\label{fig:gem_psds}
\end{figure}


\subsection{Simulated cloud observations}

A simulated observation is generated for each vertical profile in the
model test scenes. The simulations apply the same microphysics scheme
as the model, which means that they use the same six hydrometeor classes
and PSD parametrizations.

\subsubsection{Sensor configuration}
\label{sec:sensors}

Simulations of observed passive brightness temperatures are performed for the 11
highest-frequency channels of the MWI radiometer and all channels of the ICI
radiometer. The passive observations are combined with a W-band cloud radar
similar to the CloudSat Cloud Profiling Radar (CPR, \citet{stephens02,tanelli08}).

A number of simplifications are applied for the generation of the synthetic
cloud observations: Firstly, the beams of all three sensors are modeled as
perfectly coincident pencil beams. Secondly, a synthetic observation is
generated for each vertical profile from the model scenes by simulating a
one-dimensional, plane-parallel atmosphere, the properties of which are taken
from the corresponding model profile. It follows from these modeling decisions
that the atmosphere is assumed to be homogeneous across the beams of the active
and passive sensors and that they all sense the same atmospheric volume. This is
certainly not the case for space-borne observations and will incur a forward
modeling error that is not accounted for in this study. Since the focus of this
study are the fundamental synergies between the active and passive observations,
quantifying the impact of beam width and inhomogeneity is left for future
investigation.

Observations from the ICI radiometer are simulated by performing a single,
non-polarized radiative transfer simulation located at the centers of the pass
bands of each channel and averaging the resulting brightness temperatures. For
channels with multiple polarizations, only a single simulation is performed.
To compensate for this, the noise of the corresponding channel is reduced by a
factor of $\sqrt{2}$. The simulated ICI channels and assumed noise levels are
presented in  Tab.~\ref{tab:channels}. The off-nadir viewing angle of ICI
is assumed to be $48\unit{^\circ}$ at the sensor.

Observations from the MWI radiometer are simulated in a similar manner as those
from ICI. However, from MWI only channels with frequencies larger than or equal
to $89\ \unit{GHz}$ are used. The reason for this is that the footprints of the
channels with frequencies lower than $89\ \unit{GHz}$ have full-width at half
maximum of $50\ \unit{km}$ compared to only $10\ \unit{km}$ for the
higher-frequency channels. Due to the very small overlap of the footprints of
these low-frequency channels with that of the radar, it is assumed they would
not be beneficial for a synergistic retrieval and are therefore disregarded
here. The included MWI channels are listed in Tab.~\ref{tab:channels}.

\begin{table}[hbpt]
\caption{Channels of the MWI and ICI radiometers used in the retrieval.}
\label{tab:channels}
    \begin{tabular}{c|r|r}
    \multicolumn{3}{c}{MWI}\\
    Channel & Freq. [GHz] & Noise [K]\\
    \hline
    MWI-8  & $89$              & $1.1$ \\
    MWI-9  & $118.75 \pm 3.2$  & $1.3$ \\
    MWI-10 & $\pm 2.1$         & $1.3$ \\
    MWI-11 & $\pm 1.4$         & $1.3$ \\
    MWI-12 & $\pm 1.2$         & $1.3$ \\
    MWI-13 & $165.5 \pm 0.75$  & $1.3$ \\
    MWI-14 & $183.31 \pm 7.0$  & $1.2$ \\
    MWI-15 & $ \pm 6.1$        & $1.2$ \\
    MWI-16 & $ \pm 4.9$        & $1.2$ \\
    MWI-17 & $ \pm 3.4$        & $1.2$ \\
    MWI-18 & $ \pm 2.0$        & $1.3$ \\
    \end{tabular}%
    \hspace{1cm}%
    \begin{tabular}{c|r|r}
    \multicolumn{3}{c}{ICI}\\
    Channel & Freq. [GHz] & Noise [K] \\
    \hline
    ICI-1  & $183.31 \pm 7.0$ & $0.8$\\
    ICI-2  & $       \pm 3.4$ & $0.8$\\
    ICI-3  & $       \pm 2.0$ & $0.8$\\
    ICI-4  & $243    \pm 2.5$ & $\frac{1}{\sqrt{2}} \cdot 0.7$\\
    ICI-5  & $325.15 \pm 9.5$ & $1.2$\\
    ICI-6  & $       \pm 3.5$ & $1.3$\\
    ICI-7  & $       \pm 1.5$ & $1.5$\\
    ICI-8  & $448    \pm 7.2$ & $1.4$\\
    ICI-9  & $       \pm 3.0$ & $1.6$\\
    ICI-10 & $       \pm 1.4$ & $2.0$\\
    ICI-11 & $664    \pm 4.2$ & $\frac{1}{\sqrt{2}} \cdot 1.6$\\
    \end{tabular}
\end{table}

The frequency of the the cloud radar is chosen to be $94\ \unit{GHz}$ similar to
CloudSat CPR. The vertical resolution of the radar observations is assumed to be
$500\ \unit{m}$ ranging from $0.5$ to $20\ \unit{km}$ in altitude. The minimum
sensitivity is set to be $-30\ \unit{dBZ}$ and the noise at each range gate is
modeled to be independent with standard deviation $0.5\ \unit{dBZ}$. As mentioned
above, the same incidence angle as for the passive radiometers is assumed also
for the radar. In practice, this could be achieved by remapping the radar
observations to the lines of sights of the passive beams.

\subsubsection{Radiative transfer simulations}
\label{sec:orge741b86}

All simulations presented in this study were performed using Version 2.3.1245 of
the Atmospheric Radiative Transfer Simulator (ARTS, \cite{arts18}). Radar
reflectivities are computed using ARTS' built-in single-scattering radar solver.
For the simulation of passive radiances, a hybrid solver is used which combines
the DISORT \citep{disort00} scattering solver with the ARTS standard scheme for
pencil beam radiative transfer. The hybrid solver has been added to ARTS
specifically for this study and provides approximate, analytical Jacobians,
which are required for the variational retrievals of hydrometeors. All
simulations are performed assuming an ocean surface with emissivities calculated
using the Tool to Estimate Sea‐Surface Emissivity from Microwaves to
sub‐Millimeter waves (TESSEM, \cite{prigent16}). Polarization is neglected in
all simulations performed in this study. Particle scattering data are taken from
the ARTS scattering data base (hereafter ARTS SSDB, \citet{eriksson18}). Gaseous
absorption is modeled using the absorption models from \cite{rosenkranz93}
for $N_2$, $O_2$ and from \cite{rosenkranz98}  for $H_2O$.


\subsection{Retrieval algorithm}
\label{sec:orgb528563}

A one-dimensional, variational cloud retrieval algorithm is proposed to
retrieve distributions of frozen hydrometeors from the combined active and
passive observations. The algorithm uses the optimal estimation method (OEM)
developed by \cite{rodgers00}. The retrieved state $\mathbf{x} \in
   \mathbb{R}^n$ is determined by fitting a forward model $\mathbf{F} : \mathbb{R}^n
   \rightarrow \mathbb{R}^m$ to a set of observations $\mathbf{y} \in
   \mathbb{R}^m$. The best fit is determined by minimizing a cost function of
the form
\begin{align}
\mathcal{L}(\mathbf{x}, \mathbf{y}) \propto
 \left(\mathbf{F}(\mathbf{x}) - \mathbf{y} \right )^T
  \mathbf{S}_e^{-1} 
  \left ( \mathbf{F}(\mathbf{x}) - \mathbf{y} \right)
+ \left ( \mathbf{x} - \mathbf{x}_a \right )^T
 \mathbf{S}^{-1}_a 
 \left ( \mathbf{x} - \mathbf{x}_a \right ).
\end{align}
%
The cost function $\mathcal{L}(\mathbf{x}, \mathbf{y})$ corresponds to the negative
log-likelihood of the a posteriori distribution of the state $\mathbf{x}$ under
the assumptions of Gaussian a priori distribution with mean $\mathbf{x}_a$ and
covariance matrix $\mathbf{S}_a$ as well as zero-mean Gaussian measurement error
with covariance matrix $\mathbf{S}_e$.

To assess the quality of a retrieved state $\hat{\mathbf{x}}$ and corresponding simulated
observation $\hat{\mathbf{y}} = \mathbf{F}(\hat{\mathbf{x}})$, we define the following diagnostic
quantity
\begin{align}
\chi^2_y &= \delta \mathbf{y}^T
  \mathbf{S}_e^{-1} 
  \delta \mathbf{y},
\end{align}
where $\delta \mathbf{y} = \mathbf{y} - \hat{\mathbf{y}}$. The quantity
$\chi^2_y$ is here used to approximate a $\chi^2$-test for the misfit between
the observations $\mathbf{y}$ and the retrieval fit $\hat{\mathbf{y}}$. Although
a formally correct $\chi^2$-test for $\delta\mathbf{y}$ should apply a different
covariance matrix (c.f. Chapter~12 in \cite{rodgers00}), such tests were found
to yield very high values that deviate strongly from the expected chi-square
distribution. The $\chi^2_y$ value used here provides a less strict test in the
sense that it will generally be smaller than if the formally correct covariance
matrix was used.

The amount of information contained in a retrieval can be quantified by
computing the degrees of freedom for signal (DFS). Let $\mathbf{K} \in
\mathbb{R}^{m \times n}$ be the Jacobian of the forward model $\mathbf{F}$. Then
the DFS of the observations can be computed as the trace of the averaging kernel
matrix
\begin{align}
  \mathbf{A} &=(\mathbf{K}^T\mathbf{S}_e^{-1}\mathbf{K} + \mathbf{S}_a^{-1})^{-1}
  \mathbf{K}^T\mathbf{S}_e^{-1}\mathbf{K}.
\end{align}

\subsubsection{Measurement space}
\label{sec:orge7dc286}

The input for the retrieval algorithm is the combined observation vector
$\mathbf{y}$ consisting of the concatenated single-instrument observations from
the cloud radar and the two radiometers. Measurement errors are assumed to be
independent and Gaussian distributed with standard deviations according to the
noise characteristics given in Section \ref{sec:sensors}.

\subsubsection{State space}
\label{sec:method:fowardmodel}

The proposed retrieval solves for distributions of one frozen and one liquid
hydrometeor species in the atmospheric column together with profiles of
atmospheric humidity and liquid-cloud mass density. The retrieval uses the same
vertical grid as the model scenes, which have a vertical resolution of about
$500\ \unit{m}$ throughout the troposphere. If not specified otherwise,
retrieval quantities are retrieved at this resolution.

Distributions of hydrometeors in the atmospheric column are represented using
the normalized particle size distribution formalism proposed by
\cite{delanoe05}. The PSD of a hydrometeor species at a given height level is
represented by a vertical and a horizontal scaling parameter, the mass-weighted
mean diameter $D_m$ and the normalized number density $N_0^*$. Alternative 
parametrizations using mass density and $D_m$ or the mass density and $N_0^*$
have been tested but no considerable effect on retrieval performance has been
observed.

The retrieval computes vertical profiles of the two scaling parameters $D_m$ and
$N_0^*$ for each of the two hydrometeor species. The remaining shape of each PSD
is described by the shape parameters $\alpha$ and $\beta$, not to
be confused with the parameters of the mass-size relationship shown in
Tab.~\ref{tab:species_parameters}. The shape parameters are set to fixed, species-specific
values. This principle is illustrated in Fig.~\ref{fig:psds_retrieval}.
The plot displays the a-priori-assumed shapes of the particle size distribution
of frozen and liquid hydrometeors. The retrieved horizontal and vertical scaling
parameters, $D_m$ and $N_0^*$, are used as units for the axes of the plot so
that the shape of the PSD becomes independent of the retrieved mass density and
number concentration. For frozen hydrometeors, the values of the shape
parameters $\alpha$ and $\beta$ are chosen identical to version 3 of the
DARDAR-CLOUD product \citep{cazenave18}. For liquid hydrometeors, the shape
parameters are chosen so that they are equivalent to the shape used by the GEM
model for rain drops. All calculations involving particles size distributions
use the volume-equivalent diameter $D_\text{eq}$ as size variable.

\begin{figure}
\centering
\includegraphics[width = 0.5\linewidth]{figures/fig03}
\caption{PSD parametrizations for frozen and liquid hydrometeors
 used in the cloud retrieval.}
\label{fig:psds_retrieval}
\end{figure}

The temperature-dependent a priori profile for $N_0^*$ of frozen
hydrometeors is determined using the relation from \cite{delanoe14}
%
\begin{align}
N_0^* &= \exp \left ( -0.076586 \cdot (T - 272.5) + 17.948 \right ),
\end{align}
%
where $T$ is in $\unit{K}$. The a priori profile for $D_m$ for frozen
hydrometeors is chosen so that the a priori mass density is equal to
$10^{-6}\ \unit{kg\ m^{-3}}$. For liquid hydrometeors, a fixed value for $N_0^*$ of
$10^6\ \unit{m^4}$ is assumed and the a priori profile for $D_m$ is determined
similarly as for frozen hydrometeors. Values of the mass-weighted mean diameter
$D_m$ for both hydrometeor species are retrieved in linear space, whereas the
normalized number concentration parameter $N_0^*$ is retrieved in
$\text{log}_{10}$ space. As additional constraints, the retrieval of frozen
hydrometeors is restricted to the region between the freezing layer and the
tropopause, whereas the retrieval of liquid hydrometeors is restricted to below
the freezing layer.

To further regularize the retrieval, $N_0^*$ for ice is retrieved at only 10
equally-spaced grid points between freezing layer and the tropopause. Similarly,
$D_m$ and $N_0^*$ for rain are retrieved at 10 respectively 4 points between surface and
freezing layer. This was necessary to avoid the retrieval from getting stuck in
spurious local minima. An approach similar to this one is also taken in the GPM
combined precipitation retrievals \citep{grecu16}.

Humidity in the atmospheric column is retrieved in units of relative humidity at
a vertical resolution of $1\ \unit{km}$. However, instead of retrieving relative
humidity directly, an inverse hyperbolic tangens transformation is applied to
the relative humidity profile $\mathbf{\phi}$:
%
\begin{align}
x = \text{arctanh}(\frac{2 \mathbf{\phi}}{1.1} - 1.0)
\end{align}
%
The transformation restricts the retrieved relative humidity values to
the range of $[0.0, 1.1]$. The a priori profile for relative humidity
is arbitrarily chosen as
%
\begin{align}
\phi(t) = \begin{cases}
 0.7 &, 270\ \unit{K} < t \\
 0.7 - 0.01 \cdot (t - 270) & ,220 < t \leq  270\ \unit{K} \\
 0.2 \cdot (t - 270) & ,t < 220 \\
 \end{cases}.
\end{align}
%
The retrieval of liquid cloud mass density, here referred to as liquid water
content (LWC), is performed at seven equally spaced altitude levels between the
surface and the $230\ \unit{K}$ isotherm. In contrast to frozen and liquid
hydrometeors, cloud water is modeled in the retrieval forward model to be purely
absorbing using the absorption model by \cite{liebe93} for suspended liquid
cloud droplets. Liquid cloud mass density is retrieved in
$\text{log}_{10}$-space and the a priori profile is set to a fixed value of
$10^{-6}\ \unit{kg\ m^{-3}}$ in the permitted region of the atmosphere.

The a priori distributions of the 6 retrieval quantities ($N_0^*$ and $D_m$ for
frozen and liquid hydrometeors, relative humidity $\phi$, cloud water) are
assumed to be independent so that the overall a priori covariance matrix
$\mathbf{S}_a$ has block-diagonal structure. Within each block, vertical
correlations between the values of a given retrieval quantity at different
altitudes are assumed to be exponentially decaying. Hence, the correlation of
the values of retrieval quantity $q$ at points $i$ and $j$ of the retrieval grid
is computed as
%
\begin{align}
\left ( \mathbf{S}_{a,q} \right )_{i, j} &= \sigma_{q,i} \sigma_{q,j}
 \cdot \exp  \left ( -\frac{d(i, j)}{l_q} \right ),
\end{align}
%
where $\sigma_{q, i}$ is the a priori uncertainty assumed for retrieval
quantity $q$ at grid point $i$, $d(i, j)$ the distance between the grid
points and $l_q$ the quantity-specific correlation length. The assumed
a priori uncertainties and correlation lengths for the retrieval quantities
are summarized in Tab.~\ref{tab:a_priori}.

\begin{table}[h!]
\caption{A priori uncertainties and correlation
 lengths used in the retrieval.}
 \centering
\label{tab:a_priori}
    \begin{tabular}{c|r|r}
     Quantity $q$ & $\sigma_q$ & $l_q$ [km]\\
    \hline
    $\log_{10}(N_{0, \text{frozen}}^*)$    & $2$                       & $5$ \\
    $D_{m, \text{ice}}$               & $300\ \unit{\mu m}$          & $5$ \\
    $\log_{10}(N_{0, \text{liquid}}^*)$    & $2                      $ & $2$ \\
    $D_{m, \text{liquid}}$            & $500\ \unit{\mu m}$           & $2$ \\
    $\text{arctanh}(\frac{2 \cdot \phi}{1.1} - 1.0)$ & $2$       & $2$ \\
    $\log_{10}(m_\text{liquid cloud}) $ & $1$                       & $2$ \\
    \end{tabular}
\end{table}

As baselines for the assessment of the combined retrieval, also a radar-only and
a passive only-retrieval are performed. The radar-only retrieval uses the same
implementation as the combined retrieval, but only retrieves frozen and liquid
hydrometeors. For the radar-only retrieval, perfect knowledge of the atmospheric
humidity profile is assumed but liquid cloud is ignored in the retrieval forward
model.

The setup and retrieval quantities of the passive-only retrieval are similar to
the combined retrieval, with the only difference being that frozen and liquid
hydrometeors are retrieved at reduced resolution. For ice, $N_0^*$ is retrieved
at three equally spaced grid points between freezing layer and troposphere, while
$D_m$ is retrieved at five. For liquid hydormeteors, the retrieval grids for
$N_0^*$ and $D_m$ are reduced to two equally spaced points between surface and
freezing layer. Relative humidity is retrieved at a vertical resolution of
$2\ \unit{km}$.

\section{Results}
\label{sec:results}

The first part of this section presents results from a numerical experiment 
that investigates the complementary information content of the active and passive
microwave observations. Results of the combined  and the baseline retrievals applied
to the reference cloud scenes are presented in the remaining part of this section.

\subsection{Complementary information content}
\label{sec:simple_cloud}

A fundamental question regarding the benefit of combining two remote sensing
observations in a retrieval is to what extent the observations contain
non-redundant information. The degree of non-redundancy in the combined
observations is what we refer to here as complementary information content.
In order to explore this complementary information content in the radar and
radiometer observations, an idealized, homogeneous cloud layer with a thickness
of $4\ \unit{km}$ located at an altitude of $10\ \unit{km}$ in a tropical
atmosphere is considered. The cloud is assumed to consist of a single species of
frozen hydrometeors represented using the PSD parametrization which is also used
in the retrieval and described in Sec.~\ref{sec:method:fowardmodel}. As particle
model, the 8-ColumnAggregate (ID 8) from the ARTS SSDB is used.

The question that is addressed here is whether the combination of active
and passive observations is able to constrain both the horizontal and the
vertical scaling factors of the PSD of the ice particles in the cloud. To
investigate this, the $N_0^*$ and $D_m$ parameters of the homogeneous cloud
layer are varied and observations of the cloud layer are simulated. Figure
\ref{fig:contours} displays the simulated passive cloud signal, i.e. the brightness
temperature difference between clear sky and cloudy sky simulation, as filled
contours for a selection of channels of the MWI and ICI sensors. For given
values of $N_0^*$ and $D_m$, the corresponding ice mass density is given by
the relation
\begin{align}
m = \frac{\pi \rho}{4 ^ 4}N_0^* D_m^4.
\end{align}
In the figure, the cloud signal is displayed in $D_m$-mass density space and
thus shows how the measured passive cloud signal varies with the horizontal and
vertical scaling parameters of the PSD. Overlaid onto the contours of the
passive cloud signal are the isolines of the maximum radar reflectivity returned
from the cloud.

\begin{figure}
\centering
\includegraphics[width = 0.8\textwidth]{figures/fig04}
\caption{Simulated observations of a homogeneous cloud layer with
varying mass density $m$ and mass-weighted mean diameter $D_m$. The panels
display the maximum radar reflectivity in dBZ  overlaid onto the
cloud signal measured by selected radiometer channels of the MWI
(first row) and  ICI radiometers (second row).}
\label{fig:contours}
\end{figure}

The contours of the measured active and passive cloud signals show the ambiguity
of the single-instrument measurements with respect to the parameters of the PSD:
Along these contours the signal does not change, while the cloud composition
does. A necessary condition for a combined cloud retrieval to be able to resolve
this ambiguity is that the contours of the active and passive signals cross each
other. The panels in Fig.~\ref{fig:contours} thus provide an indication to what
extent the information in the radar measurement and the corresponding passive
radiometer channel provide complementary information on the parameters of the
PSD. Considering the panels corresponding to the MWI channels, the results show
that the observations contain complementary information only for very dense
clouds consisting of very large particles. In contrast to that, the ICI
observations exhibit crossing contours already at lower $m$ and $D_m$ values,
indicating that the complementary information content in these observations is
higher for less dense clouds consisting of smaller particles.

\subsection{Retrieval results}

To assess the performance of the combined cloud retrieval, the developed
algorithm has been applied to the two designated cloud scenes. The same
retrievals have been performed with a radar-only and a passive-only version of
the algorithm to serve as baselines for the combined retrieval. Each retrieval
was performed multiple times using different ice particle models. The tested
particle shapes are listed together with the corresponding mass size relations
and ARTS SSDB identifier in Tab.~\ref{tab:particles_retrieval}. Since the
results for both test scenes are qualitatively similar, not all analyses are
shown for both scenes. Instead, these are provided as a digital supplement to
this article.

\begin{table}
  \centering
  \caption{Particle model name, ARTS scattering database ID and parameters
    $\alpha, \beta$ of the mass-size relationships of the particle habits used
    in the retrieval.}
  \begin{tabular}{l|c|c|c}
    Name & ID & $\alpha$ & $\beta$ \\
    \hline
    GemCloudIce         & 31  & 440      & 3 \\
    GemSnow             & 32  & 24.0072  & 2.8571 \\
    GemGraupel          & 33  & 172.7527 & 2.9646 \\
    8-ColumnAggregate   &  8  & 65.4480  & 3      \\
    PlateType1          &  9  & 2.4770   & 0.7570 \\
    LargePlateAggregate &  20 & 2.2571   & 0.2085 \\
  \end{tabular}
  \label{tab:particles_retrieval}
\end{table}

The forward-simulated observations that were generated to test the retrievals
are shown for the first test scene in Fig.~\ref{fig:observations_a}. Independent
Gaussian noise with standard deviations according to sensor specifications has
been added to the simulated observations to account for sensor noise. It is
important to note, that the simulated observations which are used to test the
retrieval assume different microphysics than what is assumed in the retrieval:
The synthetic observations are computed using the six hydrometeor classes from
the GEM model, while the retrieval forward model assumes only two classes of
hydrometeors.

\begin{figure}
\centering
\includegraphics[width = 0.8\textwidth]{figures/fig05}
\caption{Total hydrometeor content (HMC) and simulated observations for the first test
  scene. Panel (a) displays the total hydrometeor content in the scene, i.e. the
  sum of the mass densities of all hydrometeor species of the GEM model. Panel
  (b) shows the simulated radar reflectivities. Panel (c) displays the simulated
  brightness temperatures for a selection of the channels of the MWI and ICI
  radiometers.}
\label{fig:observations_a}
\end{figure}

\subsubsection{Mass concentrations}

To provide an overview over how well the different retrieval methods are able to
reproduce the cloud structures in the test scene, the retrieved ice water
content (IWC) for the first test scene is shown in Figure \ref{fig:results_a}.
IWC is defined here as the sum of the mass densities of all frozen hydrometeor
species. This means that the reference IWC is the sum of the four frozen
hydrometeor species represented in the GEM model, whereas the retrieved IWC is
simply the mass density of the single frozen hydrometeor species assumed in the
retrieval. The results shown here were obtained using the LargePlateAggregate as
particle model, which was found to be one of the best performing particle
models.

Panel (a) of the figure displays the $\chi^2_y$ value (normalized by the
dimension of the measurement space) for each profile in the scene. A high value
of $\chi^2_y$ indicates that the retrieved state is not consistent with the
input observations. The $\chi^2_y$ value for the radar-only retrieval is
remarkably low throughout most of the scene. This may indicate that the
retrieval is insufficiently regularized, allowing it to fit the noise in the
observations. The passive-only and combined retrieval, on the contrary, have a
normalized $\chi^2_y$ value around 1 over most of the scene. Since the presented
values are normalized, the value 1 corresponds to the expected value of the
approximated chi-square distribution of $\chi^2_y$. All three retrievals exhibit
a region of elevated $\chi^2_y$ values near the core of the convective system.
In particular the high values of the passive-only and combined retrievals
indicate that the retrieval was not able to find a good fit to the observations
here.

Panel (b) displays the retrieved column-integrated IWC, the ice water path
(IWP). The IWP is given in $\unit{dB}$ relative to the reference IWP since,
owing to the high dynamic range of the reference values, the curves could
otherwise not be distinguished. Although all methods reproduce the reference
IWP fairly well, the combined retrieval yields the best overall agreement with
the reference values. Exceptions are the regions of high $\chi^2_y$ values where
the retrieval failed to find a good fit to the observations.

Panel (c) shows the IWC field retrieved using the passive-only retrieval.
Despite a certain resemblance in the overall structure between the retrieved and
reference IWC field, the results do not reproduce the vertical structure of the
cloud very well. It should be noted, however, that the displayed mass-density
range extends below the sensitivity limit of the passive-only observations
around $10^{-5}\ \unit{kg\ m^{-3}}$ (c.f. Fig. ~\ref{fig:contours}), which explains
the smeared-out appearance of the results to some extent.

The radar-only results, shown in panel (d), reproduce the vertical structure of
the cloud well. Nonetheless, when compared to the reference IWC field, certain
discrepancies are visible: The radar-only retrieval tends to overestimate the
mass density at the bottom of the cloud and underestimate the mass
concentrations at the top of the cloud.

The results of the combined retrieval are displayed in panel (e). Although some
artifacts are clearly visible in the retrieved IWC field, the retrieval
reproduces the vertical structure well. In particular, the combined retrieval
succeeds to correct some of the systematic deviations of the radar-only
retrieval: The mass density at cloud base is reduced and increased at cloud top.

\begin{figure}
\centering
\includegraphics[width = 0.8\textwidth]{figures/fig06}
\caption{Results of the ice hydrometeor retrieval for the first test scene.
  Panel (a) displays the value of the $\chi^2_y$ diagnostic normalized by the
  dimension of the measurement space of the corresponding retrieval. Panel (b)
  displays retrieved IWP in dB relative to the reference IWP. Panel (c) shows
  the reference IWC from the model scene. Panel (d), (e) and (f) display the
  retrieval results for the passive-only, radar-only and combined retrieval,
  respectively.}
\label{fig:results_a}
\end{figure}

To make the assessment of the retrieval performance more quantitative, the
reference mass concentrations are plotted against the retrieved values in
Fig.~\ref{fig:results_scatter_a_1} and \ref{fig:results_scatter_a_2}. The plots
show the results for all different retrieval configurations and tested
particle models. Markers in the plots are color-coded according to the
prevailing hydrometeor type (by mass density) in the reference scene in order
to allow assessment of the retrieval performance for the different hydrometeor
types of the GEM model.

Not surprisingly, the results from the passive-only retrieval exhibit the
strongest deviations from the diagonal. Since the passive channels alone contain
only limited information on the vertical distribution of ice in the atmosphere,
the retrieval cannot be expected to yield accurate results at the resolution
considered here. Although rather weak, a certain effect of the ice particle
model on the retrieval results can be observed. In particular, the GemCloudIce model
leads to a systematic underestimation of ice mass densities, which are less
pronounced for the other particle models.

\begin{figure}
\centering \includegraphics[width = 0.8\textwidth]{figures/fig07}
\caption{Reference IWC plotted against retrieved IWC for the tested retrieval
  configurations. Each row shows the retrieval results for the particle shape
  shown in the first panel. The following panels show the retrieval results for
  the passive only (first column), the radar only (second column) and the
  combined retrieval (third column). Markers are colored according to the
  prevailing hydrometeor type at the corresponding grid point in the test
  scene. Due to their sparsity, markers corresponding to graupel are drawn at
  twice the size of the other markers.}
\label{fig:results_scatter_a_1}
\end{figure}

The results from the radar-only retrieval are more accurate than the
passive-only retrieval, with almost all retrieval results located fairly close
to the diagonal. The most distinct feature of the radar-only results, however,
is the emergence of two clusters that extend along the diagonal but are
displaced above respectively below it. The color coding of the markers reveals
that these clusters correspond to grid points dominated by ice for the cluster
below the diagonal and snow for the cluster located above the diagonal. This
indicates that the radar-only retrieval systematically underestimates mass
densities for cloud ice but overestimates the mass density of snow. The effect
is observed for all tested particle shapes and thus likely independent of it. In
general, the radar-only results exhibit only very weak dependency on the
particle model, making the results for different particle shapes virtually
indistinguishable.

Another feature that stands out in the radar-only results is that the retrieval
does not work for graupel. This, however, can be understood by comparing the
radar reflectivities shown in Fig.~\ref{fig:observations_a} with the cloud
structure displayed in Fig.~\ref{fig:overview}. It becomes apparent that graupel
in this scene is located where the radar signal is fully attenuated. Since there
is no signal to retrieve the mass density from, this explains the bad
performance of the radar-only retrieval for these grid points.

Similar to the radar-only retrieval, the results of the combined retrieval are
located close to the diagonal. But the clusters observed in the radar-only
results are to large extent merged in the combined results. Moreover, except for
the results obtained with the GemCloudIce particle shape, the two clusters move
in closer towards the diagonal. The combined retrieval thus improves the IWC
retrieval for the specific hydrometeor species in the scene.

Nonetheless, the results for the GemCloudIce particle stand out in the results.
Even though the systematic deviations observed in the radar-only retrieval are
reduced for most particle shapes, for this specific shape they are instead
increased. The retrieval error is particularly large for snow, which is strongly
underestimated for reference mass concentrations around $10^{-4} \ \unit{kg\ m^{-3}}$.

\begin{figure}
\centering
\includegraphics[width = 0.8\textwidth]{figures/fig08}
\caption{Same as Fig.~\ref{fig:results_scatter_a_1} but for the remaining particle
  shapes.}
\label{fig:results_scatter_a_2}
\end{figure}


The results for the second test scene obtained using the LargePlateAggregate
particle model are shown in Fig.~\ref{fig:results_b}. As mentioned above, the
results are qualitatively very similar to those of the first scene. Also here,
the final OEM cost, shown in Panel~(a), displays a region of increased cost for
the passive-only and combined retrievals. This is again a region of very dense
cloud which consists of graupel and snow. Also similar to the first scene, the
passive only retrieval does not reproduce the structure of the cloud well.
Although the cloud top is placed at the right position, neither the vertical
structure of the cloud nor its base are resolved. The radar-only retrieval
resolves the vertical structure of the cloud well, but overestimates the ice
mass density in the scene. The combined retrieval also resolves the vertical
structure of the cloud well and corrects the overestimation observed in the
radar-only results to some extent.

\begin{figure}
\centering
\includegraphics[width = 0.8\textwidth]{figures/fig09}
\caption{Results of the ice hydrometeor retrieval for the second test scene.
  Panel (a) displays the value of the $\chi^2_y$ diagnostic normalized by the
  dimension of the measurement space of the corresponding retrieval. Panel (b)
  shows retrieved IWP in dB relative to the reference IWP. Panel (c) displays
  the reference mass concentrations from the model scene. Panel (d), (e) and (f)
  display the retrieval results for the passive-only, radar-only and combined
  retrieval, respectively.}
\label{fig:results_b}
\end{figure}

Scatter plots for the retrieval results from the second scene are shown in
Fig.~\ref{fig:results_scatter_b_1}. Except for the lack of cloud ice in the
scene, the results are similar to what has been observed in the first scene: The
radar-only retrieval overestimates the mass density of snow in the scene. This
effect is corrected by the combined retrieval for most of the tested particle
shapes. The exception is the GemCloudIce particle for which the retrieval of
snow particle deteriorates quite drastically.

\begin{figure}[!h]
\centering
\includegraphics[width = 0.8\textwidth]{figures/fig10}
\caption{Scatter plots of the reference and retrieved ice mass densities for
  the second test scene. The rows show the retrieval results for a given
  assumed ice particle model. The first column of each row displays a rendering
  of the particle model. The following rows display the results for the
  passive-only, the radar-only and the combined retrieval.}
\label{fig:results_scatter_b_1}
\end{figure}

To summarize retrieval performance for all tested retrieval methods and particle
shapes, the logarithmic error
\begin{align}
  \text{E}_{\text{log}_{10}} &= \log_\text{10} \left (\frac{x_\text{retrieved}}{x_\text{reference}} \right )
\end{align}
for the retrieved IWC and IWP are displayed in Fig.~\ref{fig:boxes}. The
logarithmic error in the IWC retrieval has been computed only for grid points
where either reference or retrieved IWC is larger than
$10^{-6}\ \unit{kg\ m^{-3}}$. Considering first the results of the IWC
retrieval, shown in Panel~(a) and (b), the plots confirm the findings from the
analysis above: The combined retrieval generally yields the smallest retrieval
errors. Although the spread of the retrieval errors of the radar-only retrieval
is lower in the second scene, the combined retrieval yields smaller systematic
errors.

Compared in terms of IWP, however, the results are different. Especially the
passive-only retrieval yields much lower errors for the retrieved IWP, making
the results comparable if not better than those of the other methods. For the
radar-only and combined retrievals the precision is generally increased but
systematic deviations observed in the IWC persist. This leads, particularly for
the second test scene, to significant systematic errors in the
radar-only-retrieved IWP.

In addition to this, the passive-only and the combined retrieval exhibit a
strong dependence of the retrieval error on the applied particle model.
Especially the GemCloudIce and GemSnow particle models yield large retrieval
errors for IWC and IWP. The other three particle models, however, consistently
yield smaller retrieval than the GemCloudIce and GemSnow models.


\begin{figure}[!h]
\centering
\includegraphics[width = 0.8\textwidth]{figures/fig11}
\caption{Distributions of the logarithmic retrieval error in IWC and IWP for all tested retrieval
  methods and particle shapes displayed as box plots. Colored boxes display the interquartile range (IQR)
  while whiskers show full range of all points not considered outliers. Points whose distance to
  the IQR is larger than 1.5 times the width of the IQR are considered outliers and drawn as markers.}
\label{fig:boxes}
\end{figure}

\subsubsection{Particle number densities}

Particle number densities of frozen hydrometeors have been derived from the
retrieved $N_0^*$ and $D_m$ parameters by computing the zeroth moment of the
corresponding PSD. The resulting particle number density fields are displayed
together with the reference field in Fig.~\ref{fig:results_nd_a}. To simplify
the comparison number densities are displayed only where the corresponding
reference or retrieved IWC is larger than $10^{-6}\ \unit{kg\ m^{-3}}$.

Comparing the passive-only and the radar-only retrieval to the reference field
shows that both methods have little to no skill in predicting number density
concentrations. Although the passive-only retrieval partly captures the
gradient between very high concentrations at the top of the cloud and the low
concentrations at the bottom, it is not at all resolved in the radar-only
retrieval. The combined retrieval, however, manages to reproduce this gradient
in some parts of the scene. Although its exact structure is not fully
reproduced, this clearly shows sensitivity of the retrievals to particle number
concentrations.

The combined retrieval shows the strongest deviations from the reference field
between $2$ and $3\unit{^\circ}$ latitude. Here, the results strongly
underestimate the true number concentrations. Comparison with the cloud
composition displayed in Panel (a) of Fig.~\ref{fig:overview} shows that this
region contains large amounts of both cloud ice and snow. Since the retrieval
uses only a single hydrometeor species to represent ice in the atmosphere it is
not able to represent such heterogeneous conditions. Since snow will have the
stronger impact on the observations, the retrieval in these regions tends to
predict snow rather than ice, which leads to the low retrieved number densities.

To further investigate this, Fig.~\ref{fig:results_nd_scatter_a} displays
scatter plots of the reference and retrieved number density concentrations for
all three methods and two particle models from the first test scene. Markers in
the plot are color coded according to their homogeneity in the reference scene,
here defined as the ratio of the maximum mass density of any of the frozen
hydrometeor species and total IWC. 

\begin{figure}
\centering
\includegraphics[width = 0.7\textwidth]{figures/fig12}
\caption{Reference and retrieved particle number concentrations of frozen
  hydrometeors for the first test scene obtained with the LargePlateAggregate
  particle model. Panel (a) displays the reference mass concentrations from the
  model scene. Panel (b), (c) and (d) display the retrieval results for the
  passive-only, radar-only and combined retrieval. Only values for which the corresponding
  reference or retrieved IWC was larger than  $10^{-6}\ \unit{kg\ m^{-3}}$ are shown here.}
\label{fig:results_nd_a}
\end{figure}

These results confirm that the passive-only retrieval possesses certain
sensitivity to the particle number density since the cluster at low reference
number densities corresponding to snow is placed correctly on the diagonal. The
radar-only retrieval does not exhibit any retrieval skill, hardly reproducing
any of the variation of the references values. Contrary to this, the combined
retrieval moves both clusters towards the diagonal, indicating that it is
capable of distinguishing the microphysical properties of cloud ice and snow.
Furthermore, the color coding shows that the strongest deviations between
retrieved and reference number densities occur for grid points where the cloud
composition is heterogeneous. Even for the combined retrieval, however, the
accuracy of a single retrieval value remains fairly low.

The effect of particle shape on the retrieval results is somewhat similar to
what has been observed for IWC. For the passive-only and combined retrieval, the
GemCloudIce model again yields the worst retrieval results, leading to a general
underestimation of the true particle number density. For the radar-only
retrieval no noticeable differences are observed between different particle
models. Only the results for the GemCloudIce and LargePlateAggregate particle
models are shown here since the results for the other particles are mostly similar
to those obtained with the LargePlateAggregate model.

\begin{figure}
\centering
\includegraphics[width = 0.7\textwidth]{figures/fig13}
\caption{Scatter plots of the retrieved particle number densities at grid points
  with reference mass density larger than $10^{-5}\ \unit{kg\ m^{-3}}$. Rows show
  the results for the different particle models used in the retrieval while
  column display the results for the different retrieval methods. The marker
  color encodes the homogeneity of the corresponding ice mass, which is computed
  as the ratio of the maximum mass density of any of the frozen hydrometeor
  species and total IWC.}
\label{fig:results_nd_scatter_a}
\end{figure}

\subsubsection{Information content}

The retrieval results presented above show that the combined observations allow
a more accurate retrieval of both mass and particle number density. This
confirms the experimental results from Sec.~\ref{sec:simple_cloud}, that active
and passive observations provide complementary information on the microphysics
of ice particles. The information content of the retrievals can be assessed more
quantitatively using the averaging kernel matrix. The trace of the AVK, commonly
referred to as the number of degrees of freedom for signal (DFS), quantifies the
number of independent pieces of information contained in the observations.

The distributions of the degrees of freedom of each retrieved profile in the
test scenes are displayed in Fig.~\ref{fig:dofs}. Not-surprisingly, the combined
observations exhibit the highest information content. Nevertheless, comparison
with the DFS values of the active- and passive-only retrieval shows that the
observations contain a certain degree of redundancy leading to a lower combined
DFS value than the direct sum of the two.

The grouping into retrieval quantities furthermore reveals that the largest
increase in the information content comes from water vapor, which is not
retrieved in the radar-only retrieval. Although small, a significant increase in
information content is observed for both scenes for the $N_0^*$ parameter for
ice hydrometeors. Interestingly, this increase is observed even though the
information content in the passive-only observations for $N_0^*$ is close to
zero. For the $D_m$ parameter, a small decrease is observed with respect to the
radar-only retrieval for both scenes. Since the calculation of the AVK involves
the forward model Jacobian, this effect must be related to the non-linearity of
the forward model.

\begin{figure}
\centering
\includegraphics[width = 0.7\textwidth]{figures/fig14}
\caption{Distributions of degrees of freedom of signal displayed as bar plots
  grouped by retrieval quantity and method. Results for the first test scene are
  displayed in Panel (a) and for the second test scene in Panel (b). Markers on
  the top of bars mark the extent of one standard deviation around the mean of
  each distribution.}
\label{fig:dofs}
\end{figure}

\subsubsection{Impact of assumed ice particle shape}



To further investigate the effect of the assumed ice particle shape on the
retrieval results, the mass density relations for the tested particle models are
displayed in Panel~(a) of Fig.~\ref{fig:costs}. As can be seen from this plot,
the GemCloudIce particle clearly stands out due to its large mass. Except for
the fact that the GemSnow particle does not reach down to small particle sizes,
the remaining particle models have quite similar in mass-size relations. The
extreme density of the GemCloudIce particle model for large particle sizes likely
explains the bad performance observed in the results presented above. Similarly,
the bad performance of the GemSnow model in terms of retrieved IWC and IWP is likely
due to it not covering small particle sizes.

Also displayed in Fig.~\ref{fig:costs} (panel (b) and (c)) are the $\chi^2_y$
values of the combined retrieval obtained for the tested particle models. Since
the particle shape has considerable effect on sub-millimeter observations
\citep{ekelund19}, one could hope that the retrieval results can be used to
infer the prevailing ice particle type based on the how well the retrieval can
fit the observations. Unfortunately, such clear conclusions cannot be drawn from
the results. In the first test scene, the best fit is obtained by the GemSnow,
GemCloudIce and the LargePlateAggregate particle models, although the GemSnow
and GemCloudIce models quite clearly yield the worst retrieval performance. For
the second scene, similar results are observed. Here, the GemSnow particle
consistently gives the lowest $\chi^2_y$ value but comparison with
Fig.~\ref{fig:boxes} clearly shows that it does not yield the best retrieval
performance.

\begin{figure}[!h]
\centering
\includegraphics[width = 0.8\textwidth]{figures/fig15}
\caption{Mass-size relations (Panel (a)) and $\chi^2_y$ values for the two test
  scenes (Panel (b) and (c)). The final cost curves where smoothed using a
  running average filter of a width of 20 profiles.}
\label{fig:costs}
\end{figure}


\subsubsection{Humidity and cloud water}

The developed passive and combined retrieval algorithms also retrieve profiles
of humidity and liquid cloud mass density. For relative humidity, both
retrievals demonstrate sensitivity but no improvement could be observed in the
results of the combined retrieval compared to the passive-only retrieval.
Moreover, no suitable retrieval setup was found within the scope of this study
which would yield throughout satisfactory performance. Since we do not consider
our results representative of what could be achieved with the observational
approach, they are not included here.

The liquid cloud retrieval, however, revealed an additional synergy of the radar
and passive microwave observations. The retrieval results are therefore shown in
Fig.~\ref{fig:results_cw_b} to serve as a preview for potential additional
applications of the combined retrieval approach. Panel~(a) of the figure shows
the reference and retrieved column-integrated LWC, here referred to as liquid
water path (LWP). Although the total LWC is still underestimated, the combined
observations clearly improve the LWP retrieval in all regions except those covered
by thick clouds.

Panel (b) displays the reference LWC drawn as contours on top of the total
hydrometeor content. Panel (c) and (d) show the retrieved LWC drawn on top of
the retrieved IWC for the passive-only and the combined retrieval. These results
show clearly that the combined retrieval is able to detect and retrieve liquid
clouds even when they overlap with ice clouds. Although some sensitivity of the
passive-only retrieval to LWC can be observed as well, the retrieval puts the
cloud too high in the troposphere and underestimates its LWC. This indicates
that the radar reflectivity profile contains useful information for the
retrieval to better locate cloud water in the atmospheric column.

\begin{figure}
\centering
\includegraphics[width = \textwidth]{figures/fig16}
\caption{Reference and retrieved LWC. Panel (a) shows the reference and retrieved
  LWP for each profile. Panel (b) displays reference LWC contours drawn on top
  of the total hydrometeor content. Retrieval results for passive-only and combined
  retrieval are given in Panel (c) and (d).}
\label{fig:results_cw_b}
\end{figure}


\section{Discussion}
\label{sec:discussion}

The principal aim of this study was to investigate the synergies between radar
and passive sub-millimeter observations. To this end, a simplified numerical
experiment has been presented, that qualitatively demonstrates the existence of
complementary information in the radar and passive microwave observations.
Furthermore, a combined retrieval algorithm has been developed to demonstrate
the feasibility of the synergistic retrievals and further explore their
potential as well as current limitations.

\subsection{Fundamental synergies}

The experiment presented in the first part of this study aimed to establish the
fundamental synergies of the active and passive microwave observations. It
compared the cloud signals observed by a radar, a millimeter-wave radiometer and
a sub-millimeter-wave radiometer. The results show that the combined
observations can simultaneously constrain the horizontal and vertical scaling of
the particle size distribution. However, the complementary information content
between the active and passive observations depends on both the properties of
the observed cloud and the frequency of the observations. For the lower
frequencies considered in this study, i.e. the highest channels of the MWI
radiometer, the regions where both observations provide complementary
information on the particle size distribution of the cloud are limited to very
high mass densities and particle sizes. It should be noted, however, that since
the radar simulations neglect multiple scattering, the results are likely less
accurate in this region of the cloud-parameter space. As the passive observing
frequency increases, the regions of complementary information content extend
down to smaller particle sizes and cloud mass density. Especially the
highest-frequency channels of the ICI radiometer can therefore be expected to
provide additional information on the particle size distribution of ice clouds.

\subsection{Combined cloud retrieval}

In the second part of the study, we have presented results from a combined,
variational cloud retrieval applied to synthetic observations from two test
scenes from a high-resolution atmosphere model. The results of the combined
retrieval were compared to that of a passive- and a radar-only version of the
retrieval algorithm. The simulated observations neglected potential errors
caused by different or non-overlapping antenna beams as well as inhomogeneity of
the atmosphere across the beams. On the other hand, a source of forward model
error was included by applying a more complex microphysics scheme in the
simulations than the one used in the retrieval. This allows assessing the
retrieval error caused by the simplified modeling of cloud microphysics in the
retrieval.

\subsubsection{Retrieval performance}

Of the three considered retrieval implementations, the passive-only retrieval
clearly performs worst in terms of retrieved IWC. It should be noted, however,
that the passive only retrieval presented here has not been fully optimized and
should therefore not be taken as representative of the potential performance of
the MWI and ICI radiometers for IWC retrievals. To ensure a fair comparison, the
retrieval uses almost the same a priori assumptions as the other two retrievals,
which in the presented case provide only very limited information on the
vertical structure of the cloud. As has been shown also by other studies, the
passive observations do provide information on the vertical distribution of ice
in the atmospheric column \citep{wang17, grutzun18}, but the information content
is limited to a few degrees of freedom. It is therefore unlikely that the
vertical resolution of the passive-only retrieval can be improved drastically
without further constraining it a priori, as it is typically done in retrievals
that use Monte Carlo integration or neural networks \citep{pfreundschuh18}.

With respect to IWP, however, the passive retrieval can perform as well or even
better than the radar-only and the combined retrieval. Furthermore, the results
in Figure~\ref{fig:results_nd_a} indicate that the passive observations provide
some information on the particle number concentrations, which is not the case
for the radar observations. This in itself is an interesting result as it shows
that even when considered separately, observations from active and passive
microwave sensors should be considered complementary to each other in their
information content.

As expected, the radar-only retrieval provides much better IWC retrievals than
the passive-only version. However, the results exhibit systematic deviations
from the reference values in certain regions of the cloud. The analysis of the
retrieval performance shown in Figure~\ref{fig:results_scatter_a_1},
\ref{fig:results_scatter_a_2} and \ref{fig:results_scatter_b_1} revealed that
these are caused by systematic errors in the retrieval of specific hydrometeor
species from the GEM model. A likely explanation for this is that the priori
assumptions applied in the retrieval do not fit the specific microphysical
properties of the species in the model. This hypothesis is confirmed by the
radar-retrieved number density fields shown in Fig.~\ref{fig:results_nd_a} and
Fig.~\ref{fig:results_nd_scatter_a}. While the reference distribution has two
modes corresponding to ice and snow, the retrieved values are nearly the same
throughout the whole scene. Viewed from an information content perspective, this
is plausible since the radar provides only one piece of independent information
at each range gate, which is insufficient to determine the two degrees of
freedom ($N_0^*$ and $D_m$) of the PSD. The information on the second degree of
freedom must therefore come from the a priori assumptions.

The a priori assumptions which were used in this study were similar but not
identical to what is used in the DARDAR retrievals. Also here it should be
noted, that the presented results should not be taken to be representative for
the DARDAR product. Rather than this, the DARDAR a priori settings were chosen
since they represent well established and validated assumptions for ice cloud
retrievals and therefore should provide a reasonable starting point for the
development of a combined cloud retrieval. The fact that the a priori
assumptions used in the DARDAR retrieval do not agree with the microphysical
properties of ice and snow in the GEM model, does not say much about the general
validity of these assumptions.

Despite the certain visible artifacts in the retrieved IWC field
(Fig. \ref{fig:results_a}), the analysis of the results of the combined retrieval
presented in Figs.~\ref{fig:results_scatter_a_1} , \ref{fig:results_scatter_a_2}
and in particular \ref{fig:boxes} shows that it yields, at least for most of the
tested particle models, the best retrieval performance for IWC and IWP. The
benefit of the combined observations is even more pronounced in the retrieved
number density fields (Fig.~\ref{fig:results_nd_a}). Here, the passive- and
radar-only retrieval showed little to no skill in retrieving the particle number
concentrations. The combined retrieval, however, was able to reproduce the
general structure of the number concentration fields in regions where the cloud
composition is homogeneous (Fig.~\ref{fig:results_nd_scatter_a}). In particular
this showed that the combined retrieval is able to distinguish the microphysical
properties of ice and snow in the model.

\subsubsection{Impact of the assumed particle shape}

Although the combined retrieval can reduce systematic errors in the retrieved
IWC and IWP, its performance can even degrade if an unrealistic particle habit
is used, as observed in Fig.~\ref{fig:boxes}. In general, the passive-only and
the combined retrievals display stronger sensitivity to the assumed particle shape
than the radar-only retrieval. This is plausible since the increased sensitivity
especially of the sub-millimeter radiometer channels has been highlighted
in several studies \citep{ekelund19a, fox19}.

Given the increased sensitivity of the passive-only and combined retrieval to
the assumed particle shape, it would be desirable to know which of the
properties of a particle model are most critical for its representativeness. Five
different particle models were tested here: The two most dominant from
the GEM model and three additional models taken from the ARTS SSDB. The two GEM
particles both showed the worst retrieval performance. For the GemCloudIce model, a
likely explanation for its bad performance is its very high density. The GemSnow model
has similar density as the 8-ColumnAggregate, but does not reach down to small
particle sizes, possibly explaining why it is unsuitable for the retrieval.
Nonetheless, small performance differences are observed also for the other three
models, but no clear connection to their mass-size relation can be established.
This indicates that also its specific scattering properties are important factors
that determine representativeness of a particle model.

Furthermore, it has been briefly investigated whether the goodness of the fit to
the observations can provide information on the suitability of the chosen
particle model. In particular, we aimed to address the question whether the
combined observations can constrain the dominant particle shape or whether a
good fit to the observations can be obtained regardless of the applied particle
model. Unfortunately, no evidence of a relation between the $\chi^2_y$ value and
the retrieval performance was observed. It thus remains an open question whether
and how information on the ice particle shape can be extracted from microwave
observations of ice particles.

\subsubsection{Humidity and cloud water}

As an outlook, we have also included results from the liquid cloud retrieval,
that clearly shows its capability to retrieve liquid cloud mass densities even
within mixed-phase clouds. Although certain sensitivity to cloud water is
observed also for the passive-only retrieval, the addition of the radar signal
clearly improved the localization of the cloud in the atmosphere. This explains
the observed improvement in the retrieved LWP, since at lower altitude a thicker
cloud is required to yield the same passive cloud signal. This shows that
combined radar and microwave radiometer observations can also be used for the
profiling of warm and supercooled liquid clouds.

Although no satisfactory results were obtained from the water vapor retrieval,
the retrieval results still indicate sensitivity of this setup for retrieving
atmospheric humidity. The full exploration of the potential of the combined
observations for liquid cloud and water vapor is out of the scope of this study
and is left to future investigation.

\subsubsection{Retrieval method}

The combined retrieval implementation showed robust performance on fairly
distinct and complex cloud scenes. Despite this, both scenes that were
considered here contained parts where the OEM minimization did not find a state
that results in a good fit to the observations. In contrast to that, the
radar-only retrieval did converge well in most regions where the final cost of
the combined retrieval remained high. The inability of the retrieval to fit the
observations indicates additional information that is contained in the combined
observations but which the retrieval method cannot disentangle. Furthermore, the
results exhibit visible profile-to-profile variability as well as some artifacts
in the form of high-frequency vertical oscillation. We have tried to counteract
these by increasing the vertical spatial correlation but to no avail.

This raises the question of the suitability of the OEM method applied here. The
combined retrieval violates the two fundamental assumptions of the OEM method:
The forward model is non-linear and the assumed Gaussian a priori assumptions do
not describe reality very well. In addition to that, the current implementation
of the retrieval is computationally very expensive. For further development of
the combined retrieval concept it may therefore be advisable to revisit the
applied retrieval method in search for a potentially more suitable alternative.

\subsubsection{Limitations}

Finally, it is important to consider the limitations of this study. The results
presented here are purely based on simulations and restricted to two selected
model test scenes. The validity of the presented results thus to some extent
depends on how well cloud microphysics are represented in GEM model. While this
may affect the specific performance results for the tested retrieval methods,
the main findings of this work, namely that the combined retrieval shows greater
sensitivity to the microphysical properties of ice hydrometeors than the radar-
or passive-only retrievals, should be independent of the realism of the test
scenes.

Furthermore, the forward simulations used to generate the synthetic observations
do not consider beam filling issues, assume a slightly unrealistic viewing
geometry and neglect multiple scattering in the radar simulations. For a
realistic assessment of the potential retrieval performance this should
certainly be taken into account. Again, it is important to understand the
results presented here as a study of the fundamental synergies of active and
passive microwave observations rather than an accurate performance assessment of
the combined retrieval.

\conclusions  %% \conclusions[modified heading if necessary]
\label{sec:conclusions}

The main conclusions from the results presented above are:
\begin{enumerate}
\item The complementary information in active and passive microwave observations
  can constrain two degrees of freedom of the PSD of frozen hydrometeors.
\item This reduces systematic retrieval errors for specific hydrometeor species whose
  properties are not well described by the a priori assumptions.
\item Especially the sub-millimeter channels of the ICI radiometer contribute to
  the synergistic information content for ice particles.
\end{enumerate}

In addition to this, the combined retrieval also shows improved profiling capabilities
for warm and supercooled liquid clouds.

The results presented in this study particularly highlight the complementarity
of the active and passive observations: Although the radar provides observations
at high vertical resolution, they contain insufficient information on the
microphysical properties of hydrometeors. The passive-only observations, on the
contrary, have low vertical resolution, but are more sensitive to cloud
microphysics allowing a potentially more accurate IWP retrieval than what can be
obtained from the radar alone. A synergistic retrieval using both types of
observations allows combining the high vertical resolution of the radar with the
sensitivity to cloud microphysics of the passive observations, which
yields more accurate retrievals of IWC, IWP and particle number densities.

Synergistic retrievals from active and passive microwave observations ideally
complement currently available observation systems that combine radar with
observations in the visible or infrared. The advantage of combined microwave
observations is that they provide sensitivity throughout the whole cloud, where
visible and infrared observations would be saturated. Where only information
from the radar is available, a retrieval based on optical or infrared
observations has to rely on a priori assumptions, which may cause similar
systematic errors as what has been observed in this study. In addition to this,
our results underline the benefits of ICI's sub-millimeter channels, which
significantly improve the sensitivity of the passive observations to smaller
particle sizes and mass densities and thus narrow the sensitivity gap between
the observing frequencies of traditional microwave imagers and observations in
the infrared and visible domain.

The upcoming launch of the ICI and MWI radiometers thus provides a great
opportunity for a potential synergistic cloud-radar missions. Such a mission
would have a unique scientific value for the study of frozen hydrometeors
because of its ability to better determine the microphysical properties of
hydrometeors even inside of thick clouds. Since such information is currently
simply not available at a global scale, such a mission would be valuable not
only in itself but also for other earth observation systems by establishing more
reliable a priori assumptions on cloud microphysics.

The results presented in this study not only show the potential of the combined
retrieval approach but also demonstrate its feasibility. Although further work
will be required to fully understand the effect of particle shape and PSD, the
concept is mature enough to be applied to real observations. Since airborne
demonstrators of sub-millimeter radiometers are available already today, the
combined retrievals could be applied in future field campaigns to study ice
cloud microphysics.

Overall, the combined active and passive microwave retrievals are a promising
concept that deserves further exploration. Regardless whether airborne or
spaceborne, combined active and passive microwave observations have great
potential to improve the understanding of the microphysical properties of ice
hydrometeors.

%% The following commands are for the statements about the availability of data sets and/or software code corresponding to the manuscript.
%% It is strongly recommended to make use of these sections in case data sets and/or software code have been part of your research the article is based on.

\codeavailability{All code used to produce the results in this study is publicly available online. \citep{mcrf}.} %% use this section when having only software code available

\dataavailability{Data to reproduce the simulations leading to the presented results will
  be made available on request.}

%% Regarding figures and tables in appendices, the following two options are possible depending on your general handling of figures and tables in the manuscript environment:

%% Option 1: If you sorted all figures and tables into the sections of the text, please also sort the appendix figures and appendix tables into the respective appendix sections.
%% They will be correctly named automatically.

%% Option 2: If you put all figures after the reference list, please insert appendix tables and figures after the normal tables and figures.
%% To rename them correctly to A1, A2, etc., please add the following commands in front of them:

\appendixfigures  %% needs to be added in front of appendix figures

\appendixtables   %% needs to be added in front of appendix tables

%% Please add \clearpage between each table and/or figure. Further guidelines on figures and tables can be found below.



\authorcontribution{Simon Pfreundschuh has implemented the retrieval, performed
  the data analysis and written the manuscript. Patrick Eriksson and Richard
  Larsson have added code to the ARTS radiative transfer model that was required
  to perform the presented calculations. Stefan A. Buehler, Patrick Eriksson,
  Manfred Brath and Simon Pfreundschuh have collaborated on the study that lead
  to the results presented here. David Duncan and Robin Ekelund have contributed
  to the conceptualization of the study through comments and advice.}



\competinginterests{No competing interests are present.} %% this section is mandatory even if you declare that no competing interests are present

\begin{acknowledgements}

The combined and radar-only were developed as part of the ESA-funded study
``Scientific Concept Study for Wide-Swath High-Resolution Cloud Profiling''
(Contract number: 4000119850/17/NL/LvH). The authors would like to thank
study manager Tobias Wehr for his valuable input and guidance.

Furthermore, the authors would like to acknowledge the work of Zhipeng Qu,
Howard Barker, and Jason Cole from Environment and Climate Change Canada who
produced the model scenes that were used to test the retrieval.

The work of SP, PE and RE on this study was financially supported by the Swedish National Space Agency
(SNSA) under grants 150/14 and 166/18.

SB was supported by the Deutsche Forschungsgemeinschaft (DFG, German Research
Foundation) under Germany's Excellence Strategy --- EXC 2037 'Climate, Climatic
Change, and Society' --- Project Number: 390683824, contributing to the Center
for Earth System Research and Sustainability (CEN) of Universit\"{a}t Hamburg.

The computations for this study were performed using several freely available programming
languages and software packages, most prominently the Python language
\citep{python}, the IPython computing environment \citep{ipython}, the numpy
package for numerical computing \citep{numpy} and matplotlib for generating
figures \citep{matplotlib}.

\end{acknowledgements}




%% REFERENCES

%% The reference list is compiled as follows:


%% Since the Copernicus LaTeX package includes the BibTeX style file copernicus.bst,
%% authors experienced with BibTeX only have to include the following two lines:

\bibliographystyle{copernicus}
\bibliography{references}
%%
%% URLs and DOIs can be entered in your BibTeX file as:
%%
%% URL = {http://www.xyz.org/~jones/idx_g.htm}
%% DOI = {10.5194/xyz}


%% LITERATURE CITATIONS
%%
%% command                        & example result
%% \citet{jones90}|               & Jones et al. (1990)
%% \citep{jones90}|               & (Jones et al., 1990)
%% \citep{jones90,jones93}|       & (Jones et al., 1990, 1993)
%% \citep[p.~32]{jones90}|        & (Jones et al., 1990, p.~32)
%% \citep[e.g.,][]{jones90}|      & (e.g., Jones et al., 1990)
%% \citep[e.g.,][p.~32]{jones90}| & (e.g., Jones et al., 1990, p.~32)
%% \citeauthor{jones90}|          & Jones et al.
%% \citeyear{jones90}|            & 1990



%% FIGURES

%% When figures and tables are placed at the end of the MS (article in one-column style), please add \clearpage
%% between bibliography and first table and/or figure as well as between each table and/or figure.


%% ONE-COLUMN FIGURES

%%f
%\begin{figure}[t]
%\includegraphics[width=8.3cm]{FILE NAME}
%\caption{TEXT}
%\end{figure}
%
%%% TWO-COLUMN FIGURES
%
%%f
%\begin{figure*}[t]
%\includegraphics[width=12cm]{FILE NAME}
%\caption{TEXT}
%\end{figure*}
%
%
%%% TABLES
%%%
%%% The different columns must be seperated with a & command and should
%%% end with \\ to identify the column brake.
%
%%% ONE-COLUMN TABLE
%
%%t
%\begin{table}[t]
%\caption{TEXT}
%\begin{tabular}{column = lcr}
%\tophline
%
%\middlehline
%
%\bottomhline
%\end{tabular}
%\belowtable{} % Table Footnotes
%\end{table}
%
%%% TWO-COLUMN TABLE
%
%%t
%\begin{table*}[t]
%\caption{TEXT}
%\begin{tabular}{column = lcr}
%\tophline
%
%\middlehline
%
%\bottomhline
%\end{tabular}
%\belowtable{} % Table Footnotes
%\end{table*}
%
%%% LANDSCAPE TABLE
%
%%t
%\begin{sidewaystable*}[t]
%\caption{TEXT}
%\begin{tabular}{column = lcr}
%\tophline
%
%\middlehline
%
%\bottomhline
%\end{tabular}
%\belowtable{} % Table Footnotes
%\end{sidewaystable*}
%
%
%%% MATHEMATICAL EXPRESSIONS
%
%%% All papers typeset by Copernicus Publications follow the math typesetting regulations
%%% given by the IUPAC Green Book (IUPAC: Quantities, Units and Symbols in Physical Chemistry,
%%% 2nd Edn., Blackwell Science, available at: http://old.iupac.org/publications/books/gbook/green_book_2ed.pdf, 1993).
%%%
%%% Physical quantities/variables are typeset in italic font (t for time, T for Temperature)
%%% Indices which are not defined are typeset in italic font (x, y, z, a, b, c)
%%% Items/objects which are defined are typeset in roman font (Car A, Car B)
%%% Descriptions/specifications which are defined by itself are typeset in roman font (abs, rel, ref, tot, net, ice)
%%% Abbreviations from 2 letters are typeset in roman font (RH, LAI)
%%% Vectors are identified in bold italic font using \vec{x}
%%% Matrices are identified in bold roman font
%%% Multiplication signs are typeset using the LaTeX commands \times (for vector products, grids, and exponential notations) or \cdot
%%% The character * should not be applied as mutliplication sign
%
%
%%% EQUATIONS
%
%%% Single-row equation
%
%\begin{equation}
%
%\end{equation}
%
%%% Multiline equation
%
%\begin{align}
%& 3 + 5 = 8\\
%& 3 + 5 = 8\\
%& 3 + 5 = 8
%\end{align}
%
%
%%% MATRICES
%
%\begin{matrix}
%x & y & z\\
%x & y & z\\
%x & y & z\\
%\end{matrix}
%
%
%%% ALGORITHM
%
%\begin{algorithm}
%\caption{...}
%\label{a1}
%\begin{algorithmic}
%...
%\end{algorithmic}
%\end{algorithm}
%
%
%%% CHEMICAL FORMULAS AND REACTIONS
%
%%% For formulas embedded in the text, please use \chem{}
%
%%% The reaction environment creates labels including the letter R, i.e. (R1), (R2), etc.
%
%\begin{reaction}
%%% \rightarrow should be used for normal (one-way) chemical reactions
%%% \rightleftharpoons should be used for equilibria
%%% \leftrightarrow should be used for resonance structures
%\end{reaction}
%
%
%%% PHYSICAL UNITS
%%%
%%% Please use \unit{} and apply the exponential notation


\end{document}
