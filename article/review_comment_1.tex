% Created 2018-06-25 Mo 13:07
% Intended LaTeX compiler: pdflatex
\documentclass[11pt]{scrartcl}
\usepackage[utf8]{inputenc}
\usepackage[T1]{fontenc}
\usepackage{graphicx}
\usepackage{grffile}
\usepackage{longtable}
\usepackage{wrapfig}
\usepackage{rotating}
\usepackage[normalem]{ulem}
\usepackage{amsmath}
\usepackage{textcomp}
\usepackage{amssymb}
\usepackage{capt-of}
\usepackage{hyperref}
\author{simon}
\date{\today}
\title{}
\hypersetup{
 pdfauthor={Simon Pfreundschuh},
 pdftitle={},
 pdfkeywords={},
 pdfsubject={},
 pdfcreator={Emacs 24.5.1}, 
 pdflang={English}}
\begin{document}

\setlength{\parindent}{0cm}

\section*{General comments}

\subsection{Referee comment:}

1. As noted in Section 4.2.4, the a priori assumptions do not describe reality
very well. In particular, I suspect that the information content of Dm and N0* is
highly dependent on the a priori assumptions of these two variables in the
retrieval framework. Especially with a radar measurement, since Z is sensitive to
both parameters over a wide range of the parameter space, the relative
sensitivity and therefore information content willalmost entirely depend on the
relative constraints on these parameters imposed by Xaand Sa. As such it is
imperative to accurately characterize these. I understand the choice to use the
DARDAR constraints, but it’s clear from the cross-section plots thatthe model
ice particle concentrations vary over a much wider range than the roughly2
orders of magnitude that Eq. 4 provides over a 220-272 K temperature range.
So, when the retrieval results are compared to model “reality”, it seems that a
lot of N0* variability is folded into Dm and this is especially evident in
Figures 13 and 14. My overall concern is that it is difficult to interpret some
of the results when the modelfields and the a priori assumptions differ so
strongly.

\subsubsection*{Author response:}

It should be clarified here that Eq. (4) only gives the variation of the mean
profile of the a priori for $N_0^*$ and that the standard deviation for $N_0^*$
has been set to a value of $2$ in log-space, allowing $N_0^*$ to vary over several
orders of magnitude.

Nonetheless, the point raised by the referee remains valid. Since the topic
of the study are synergies between radar and radiometer observations, we aimed
to keep the a priori assumptions realistic (the DARDAR mean profile) but at the
same time sufficiently loose (high std. dev.) in order to not introduce information
a priori that may be provided by the synergy. To clarify this, the following
lines haves been added in the introduction:

\vspace{1em}

\textit{The standard deviations in the covariance matrix have been deliberately chosen
  to be very loose in order to not synthetically introduce information that may be
  provided by the radiometer observations.}

\subsection*{Referee comment}

2. Forward model error is introduced when the different species present in the
model microphysics are combined into one species and when different scattering
models are used to represent the ice particles. That this is not represented in
Se could lead to over-fitting and poor convergence (I suspect this is part of
the reason why the normalized cost is much higher for the radiometer-including
retrievals). It should be relatively easy to quantify this error by re-running
the simulations with the retrieval assumptions(combining ice species, different
scattering models), and I suspect that this error term would dominate the
instrument noise term for many channels.

\subsubsection*{Author response}

This is certainly another valid point and has been addressed in the way suggested
by the authors. All calculations were repeated and the corresponding results
updated in the manuscript. To clarify this a sub-section specifying the assumed observation
errors has been added to Sect. 2.3..
\vspace{1em}

\textit{The standard deviations in the covariance matrix have been deliberately chosen
  to be very loose in order to not synthetically introduce information that may be
  provided by the radiometer observations.}

\subsection*{Referee comment}
I am quite certain that neither “Gaussianity” (p.3, line 2) nor “overproportionally” (p. 12,
line 20) are real words.

\end{document}
